\documentclass[9pt,twoside,lineno]{pnas-new}
% Use the lineno option to display guide line numbers if required.

\templatetype{pnassupportinginfo}
% \readytosubmit %% Uncomment this line before submitting, so that the instruction page is removed.

\graphicspath{{figures/}}
\DeclareMathOperator{\Tr}{Tr}

\title{Learning complex models with invertible neural networks: a likelihood-free Bayesian approach}
\author{Stefan T. Radev, Ulf K. Mertens, Andreas Voss, Lynton Ardizzone, and Ullrich Köthe}
\correspondingauthor{Stefan Radev.\\E-mail: stefan.radev@psychologie.uni-heidelberg.de}

\begin{document}


\maketitle

%% Adds the main heading for the SI text. Comment out this line if you do not have any supporting information text.
\SItext


\section*{Results}


\subsection*{Performance metrics}

In the following, the computation of the performance metrics used throughout the main text is detailed.

\subsubsection*{Normalized Root Mean Squared Error}
The normalized root mean squared error (NRMSE) between a sample of true parameters $\{\theta^{(i)}\}_{i=1}^{n}$ and a sample of estimated parameters $\{\hat{\theta}^{(i)}\}_{i=1}^{n}$ is given by:
\begin{align*}
NRMSE = \sqrt{\sum_{i=1}^{n}\frac{\left(\theta^{(i)}-\hat{\theta}^{(i)}\right)^{2}}{\theta_{max}- \theta_{min}}} \numberthis \label{eqn:1}
\end{align*}
Due to the normalization factor $\theta_{max}-\theta_{min}$, the NRMSE is scale-independent, and thus suitable for comparing the recovery across parameters having different numerical ranges. The NRMSE is zero when the estimates are exactly equal to the true values.
\subsubsection*{Coefficient of Determination }
The coefficient of determination $R^{2}$ gives the proportion of variance in a sample of true parameters $\{\theta^{(i)}\}_{i=1}^{n}$ that is "explained" by a sample of estimated parameters $\{\hat{\theta}^{(i)}\}_{i=1}^{n}$. It is computed as:
\begin{align*}
R^{2} = 1 - \sum_{i=1}^{n}\frac{\left(\theta^{(i)}-\hat{\theta}^{(i)}\right)^{2}}{\left(\theta^{(i)}-\bar{\theta}^{(i)}\right)^{2}} \numberthis \label{eqn:2}
\end{align*}
where $\bar{\theta}$ denotes the mean of the true parameter samples. When $R^{2}$ equals $1$, it means that the estimates are perfect reconstructions of the true parameters.

\subsubsection*{Kullback-Leibler Divergence} The Kullback-Leibler divergence ($D_{KL}$) quantifies the increase in entropy incurred by approximating a target probability distribution $P$ with a distribution $Q$. Its general form for absolutely continuous distributions is given by
\begin{align*}
D_{KL}(P||Q) = \int_{-\infty}^{\infty} p(x)\log\frac{p(x)}{q(x)} dx \numberthis \label{eqn:3}
\end{align*}
where $p$ and $q$ denote the pdfs of $P$ and $Q$. In the case where $P$ and $Q$ are both multivariate Gaussian distributions, the KL divergence can be computed in closed form \cite{hershey2007approximating}:
\begin{align*}
D_{KL}(P||Q) = \frac{1}{2}\left[\log\frac{\det\boldsymbol{\Sigma}_{q}}{\det\boldsymbol{\Sigma}_{p}} + \Tr(\boldsymbol{\Sigma}_{q}^{-1}\boldsymbol{\Sigma}_{p}) - d + (\boldsymbol{\mu}_{p} - \boldsymbol{\mu}_{q})^{T}\boldsymbol{\Sigma}_{q}^{-1}(\boldsymbol{\mu}_{p} - \boldsymbol{\mu}_{q})\right] \numberthis \label{eqn:4}
\end{align*}
where $\boldsymbol{\Sigma}_{p}$ and $\boldsymbol{\Sigma}_{q}$ denote the covariance matrices of $p$ and $q$, $\boldsymbol{\mu}_{p}$ and $\boldsymbol{\mu}_{q}$ the respective mean vectors, and $d$ the number of dimensions of the Gaussian. In the case of diagonal Gaussian distributions, Eq.\ref{eqn:4} reduces to:
\begin{align*}
D_{KL}(P||Q) = \sum_{i=1}^d\left(\log\frac{\sigma_{q,i}}{\sigma_{p,i}} + \frac{\sigma_{p,i}^{2} + (\mu_{q,i} - \mu_{p,i})^{2}}{2\sigma_{q,i}^{2}} - \frac{1}{2} \right) \numberthis \label{eqn:5}
\end{align*}
Even though the KL divergence is not a proper distance metric, as it is not symmetric in its arguments, it can be used to quantify the error of approximation and serve as a metric for comparing different methods. 

\subsubsection*{Simulation-Based Calibration}
Simulation-based calibration is a recently proposed method for validating the accuracy of posterior samples generated by a Bayesian sampling method \cite{talts2018validating}. It is based on the so called \textit{self-consistency} of the Bayesian joint distribution. Given a sample from the prior distribution $\tilde{\theta} \sim p(\theta)$ and a sample from the data-generating process $\tilde{x} \sim p(x|\tilde{\theta})$, one can integrate $\tilde{\theta}$ and $\tilde{x}$ out of the Bayesian joint distribution to recover back the prior of $\theta$:
\begin{align*}
p(\theta) &= \int p(\theta,\tilde{\theta},\tilde{x})d\tilde{x}d\tilde{\theta} \numberthis  \label{eqn:6} \\
&= \int p(\theta,\tilde{x}|\tilde{\theta})p(\tilde{\theta})d\tilde{x}d\tilde{\theta} \numberthis \label{eqn:7} \\
&= \int p(\theta|\tilde{x})p(\tilde{x}|\tilde{\theta})p(\tilde{\theta})d\tilde{x}d\tilde{\theta} \numberthis \label{eqn:8}
\end{align*}
If the Bayesian sampling method produces samples from the exact posterior, the equality implied by Eq.\ref{eqn:8} should hold regardless of the particular form of the posterior. Thus, any violation of this equality indicates some error incurred by the sampling method. The authors in \cite{talts2018validating} propose the following algorithm based on rank-statistics for detecting such violations.
\begin{figure}
\centering
\includegraphics[width=\textwidth]{acb.png}
\caption{Second figure}
\end{figure}


%%% Add this line AFTER all your figures and tables
\FloatBarrier

\movie{Type caption for the movie here.}

\movie{Type caption for the other movie here. Adding longer text to show what happens, to decide on alignment and/or indentations.}

\movie{A third movie, just for kicks.}

\dataset{dataset_one.txt}{Type or paste caption here.}

\dataset{dataset_two.txt}{Type or paste caption here. Adding longer text to show what happens, to decide on alignment and/or indentations for multi-line or paragraph captions.}

\bibliography{references}

\end{document}